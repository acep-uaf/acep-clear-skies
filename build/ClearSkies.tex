% Options for packages loaded elsewhere
\PassOptionsToPackage{unicode}{hyperref}
\PassOptionsToPackage{hyphens}{url}
\PassOptionsToPackage{dvipsnames,svgnames,x11names}{xcolor}
%
\documentclass[
  12pt,
]{article}
\usepackage{amsmath,amssymb}
\usepackage{lmodern}
\usepackage{iftex}
\ifPDFTeX
  \usepackage[T1]{fontenc}
  \usepackage[utf8]{inputenc}
  \usepackage{textcomp} % provide euro and other symbols
\else % if luatex or xetex
  \usepackage{unicode-math}
  \defaultfontfeatures{Scale=MatchLowercase}
  \defaultfontfeatures[\rmfamily]{Ligatures=TeX,Scale=1}
\fi
% Use upquote if available, for straight quotes in verbatim environments
\IfFileExists{upquote.sty}{\usepackage{upquote}}{}
\IfFileExists{microtype.sty}{% use microtype if available
  \usepackage[]{microtype}
  \UseMicrotypeSet[protrusion]{basicmath} % disable protrusion for tt fonts
}{}
\makeatletter
\@ifundefined{KOMAClassName}{% if non-KOMA class
  \IfFileExists{parskip.sty}{%
    \usepackage{parskip}
  }{% else
    \setlength{\parindent}{0pt}
    \setlength{\parskip}{6pt plus 2pt minus 1pt}}
}{% if KOMA class
  \KOMAoptions{parskip=half}}
\makeatother
\usepackage{xcolor}
\IfFileExists{xurl.sty}{\usepackage{xurl}}{} % add URL line breaks if available
\IfFileExists{bookmark.sty}{\usepackage{bookmark}}{\usepackage{hyperref}}
\hypersetup{
  colorlinks=true,
  linkcolor={blue},
  filecolor={Maroon},
  citecolor={Blue},
  urlcolor={Blue},
  pdfcreator={LaTeX via pandoc}}
\urlstyle{same} % disable monospaced font for URLs
\usepackage[margin=1in]{geometry}
\usepackage{longtable,booktabs,array}
\usepackage{calc} % for calculating minipage widths
% Correct order of tables after \paragraph or \subparagraph
\usepackage{etoolbox}
\makeatletter
\patchcmd\longtable{\par}{\if@noskipsec\mbox{}\fi\par}{}{}
\makeatother
% Allow footnotes in longtable head/foot
\IfFileExists{footnotehyper.sty}{\usepackage{footnotehyper}}{\usepackage{footnote}}
\makesavenoteenv{longtable}
\usepackage{graphicx}
\makeatletter
\def\maxwidth{\ifdim\Gin@nat@width>\linewidth\linewidth\else\Gin@nat@width\fi}
\def\maxheight{\ifdim\Gin@nat@height>\textheight\textheight\else\Gin@nat@height\fi}
\makeatother
% Scale images if necessary, so that they will not overflow the page
% margins by default, and it is still possible to overwrite the defaults
% using explicit options in \includegraphics[width, height, ...]{}
\setkeys{Gin}{width=\maxwidth,height=\maxheight,keepaspectratio}
% Set default figure placement to htbp
\makeatletter
\def\fps@figure{htbp}
\makeatother
\setlength{\emergencystretch}{3em} % prevent overfull lines
\providecommand{\tightlist}{%
  \setlength{\itemsep}{0pt}\setlength{\parskip}{0pt}}
\setcounter{secnumdepth}{5}
\newlength{\cslhangindent}
\setlength{\cslhangindent}{1.5em}
\newlength{\csllabelwidth}
\setlength{\csllabelwidth}{3em}
\newlength{\cslentryspacingunit} % times entry-spacing
\setlength{\cslentryspacingunit}{\parskip}
\newenvironment{CSLReferences}[2] % #1 hanging-ident, #2 entry spacing
 {% don't indent paragraphs
  \setlength{\parindent}{0pt}
  % turn on hanging indent if param 1 is 1
  \ifodd #1
  \let\oldpar\par
  \def\par{\hangindent=\cslhangindent\oldpar}
  \fi
  % set entry spacing
  \setlength{\parskip}{#2\cslentryspacingunit}
 }%
 {}
\usepackage{calc}
\newcommand{\CSLBlock}[1]{#1\hfill\break}
\newcommand{\CSLLeftMargin}[1]{\parbox[t]{\csllabelwidth}{#1}}
\newcommand{\CSLRightInline}[1]{\parbox[t]{\linewidth - \csllabelwidth}{#1}\break}
\newcommand{\CSLIndent}[1]{\hspace{\cslhangindent}#1}
\usepackage{etoolbox}
\usepackage[most]{tcolorbox}
\usepackage{graphicx}
\usepackage{xparse}
\tcbuselibrary{breakable}

% --- Disable floating figures globally ---
\usepackage{float}
\let\origfigure\figure
\let\endorigfigure\endfigure
\renewenvironment{figure}[1][H]{\origfigure[H]}{\endorigfigure}

% --- Colors ---
\usepackage{xcolor}
\definecolor{speculativeback}{RGB}{247,232,255}  % #f7e8ff
\definecolor{speculativeframe}{RGB}{139,43,226}  % #8b2be2

% --- Left-align all Pandoc tables ---
\usepackage{longtable}
\usepackage{etoolbox}
\makeatletter
\patchcmd\longtable{\par}{\par\raggedright}{}{}
\setlength{\LTleft}{0pt}
\setlength{\LTright}{0pt}
\makeatother


% Helper to show a blank title if empty
\newcommand{\blanktitle}{\mbox{}}

% =========================================================
% Concept Boxes
% =========================================================
\newtcolorbox{infobox}[1][Information]{%
  title=\ifstrempty{#1}{\blanktitle}{#1},
  colback=blue!3!white,
  colframe=blue!50!black,
  colbacktitle=blue!10!white,
  fonttitle=\bfseries,
  coltitle=black,
  enhanced, sharp corners, breakable,
  after=\par\vspace{6pt}
}

\newtcolorbox{proposedbox}[1][Proposed Concept]{%
  title=\ifstrempty{#1}{\blanktitle}{#1},
  colback=blue!5!white,
  colframe=blue!75!black,
  colbacktitle=blue!15!white,
  fonttitle=\bfseries,
  coltitle=black,
  enhanced, sharp corners, breakable,
  after=\par\vspace{6pt}
}

\newtcolorbox{establishedbox}[1][Established Concept]{%
  title=\ifstrempty{#1}{\blanktitle}{#1},
  colback=green!5!white,
  colframe=green!75!black,
  colbacktitle=green!15!white,
  fonttitle=\bfseries,
  coltitle=black,
  enhanced, sharp corners, breakable,
  after=\par\vspace{6pt}
}

\newtcolorbox{speculativebox}[1][Speculative Concept]{%
  title=\ifstrempty{#1}{\blanktitle}{#1},
  colback=speculativeback,
  colframe=speculativeframe,
  colbacktitle=speculativeback!80!white, % lighter variant for title bar
  fonttitle=\bfseries,
  coltitle=black,
  enhanced, sharp corners, breakable,
  after=\par\vspace{6pt}
}


\newtcolorbox{cautionbox}[1][Caution]{%
  title=\ifstrempty{#1}{\blanktitle}{#1},
  colback=orange!5!white,
  colframe=orange!75!black,
  colbacktitle=orange!15!white,
  fonttitle=\bfseries,
  coltitle=black,
  enhanced, sharp corners, breakable,
  after=\par\vspace{6pt}
}

\newtcolorbox{warningbox}[1][Warning]{%
  title=\ifstrempty{#1}{\blanktitle}{#1},
  colback=orange!5!white,
  colframe=orange!75!black,
  colbacktitle=orange!15!white,
  fonttitle=\bfseries,
  coltitle=black,
  enhanced, sharp corners, breakable,
  after=\par\vspace{6pt}
}

\newtcolorbox{dangerbox}[1][Danger]{%
  title=\ifstrempty{#1}{\blanktitle}{#1},
  colback=red!5!white,
  colframe=red!75!black,
  colbacktitle=red!15!white,
  fonttitle=\bfseries,
  coltitle=black,
  enhanced, sharp corners, breakable,
  after=\par\vspace{6pt}
}
\ifLuaTeX
  \usepackage{selnolig}  % disable illegal ligatures
\fi

\author{}
\date{}

\begin{document}

\renewcommand*\contentsname{Contents}
{
\hypersetup{linkcolor=}
\setcounter{tocdepth}{3}
\tableofcontents
}
\hypertarget{executive-summary}{%
\section{Executive Summary}\label{executive-summary}}

Alaska has the world's highest concentration of islanded
microgrids---small, self-contained power systems that sustain hundreds
of remote communities unconnected to any central grid. These systems are
lifelines, but they face growing challenges: rising fuel and logistics
costs, access to technical talent, fragile Internet connectivity, and
increasing exposure to cyber threats as digital controls expand into
every corner of local infrastructure.

\textbf{Clear Skies} proposes a new path forward --- a
\emph{local-first, Internet independent cyberinfrastructure reference
architecture} designed specifically with resiliency and security in mind
for Alaska's rural communities. By establishing scalable,
community-owned data and control systems, Clear Skies enables essential
digital services---SCADA, communications, cybersecurity, and data
management---to operate \textbf{independently of Internet access}.

The Clear Skies reference architecture defines a layered and modular
approach to resilience:

\begin{itemize}
\tightlist
\item
  \textbf{Layer 0 -- Hardware Foundations:} three tiers of scalable
  deployment (Camp, Village, Regional) that balance cost, capacity, and
  redundancy using commodity or enterprise hardware.
\item
  \textbf{Layer 1 -- Cyberinfrastructure (CI):} a local Software-Defined
  Data Center (SDDC) providing virtualization, networking, storage, and
  identity services built on open-source tools such as Proxmox, Ceph,
  and OPNsense.
\item
  \textbf{Layer 2 -- Local Services:} modular applications for
  operational technology (OT), industrial IoT, emergency communications,
  and community collaboration---entirely hosted and managed within the
  local network.
\item
  \textbf{Layer 3 -- Community Connections:} secure, zero-trust bridges
  that enable inter-village collaboration, cross-site data sharing, and
  regional coordination while preserving digital sovereignty.
\end{itemize}

Clear Skies is more than an IT architecture---it is an \textbf{enabler
of digital sovereignty}, extending the principles of local
self-determination into the digital domain.\\
Its modular design allows each community to start small, learn, and
grow---building capacity, reducing dependency, and cultivating a
workforce skilled in managing their own resilient, secure digital
infrastructure.

The Clear Skies reference architecture strategically supports diverse
use cases---Home Labs, Test Beds, Training Grounds, On-site CI,
Community CI, Emergency Communications, and Industrial CI---providing a
roadmap for communities not just to \emph{keep the lights on}, but to
illuminate their own path toward independence, innovation, and long-term
sustainability in the digital era.

\hypertarget{introduction}{%
\section{Introduction}\label{introduction}}

\hypertarget{vision-statement}{%
\subsubsection{Vision Statement}\label{vision-statement}}

\textbf{Clear Skies} is a locally grown initiative to build
\textbf{community-owned, cloud-free digital infrastructure} across rural
Alaskan microgrid communities. It empowers villages, tribes, and
regional utilities to host and secure their own data, communications,
and operational systems --- right where they live and work without
reliance on distant cloud services.

By bringing computing power, cybersecurity, and communications back
under local control, \textbf{Clear Skies} advances \emph{digital
sovereignty} as as a modern expression of community and tribal
self-determination.\\
It strengthens self-reliance, ensures continuity during network outages,
and creates a foundation for innovation that reflects Alaska's values of
\textbf{independence, stewardship}, and \textbf{cooperation}.

The following reference architecture outlines how Clear Skies can be
implemented in scalable layers, from physical infrastructure to regional
collaboration.

\hypertarget{problem-statement}{%
\subsection{Problem Statement}\label{problem-statement}}

Alaska has the worlds highest concentration of island-ed micro-grids in
the world. The remote communities are not connected by roads or
transmission lines. Most generate power primarily with diesel, and the
fuel is expensive, especially if the community is not on a the coast or
river systems where fuel can be barged in. For those remote communities
fuel must be flown in.

Internet access in these communities is also a constrained resource.
Some coastal communities have access to high speed fiber optic
connections, while others have been limited to expensive geosynchronous
satellite communications. Though in 2 of the last 3 years, sea ice has
cut burred cables resulting several month service outages. Low earth
orbit (LEO)(\protect\hyperlink{ref-LowEarthOrbit2025}{{``Low {Earth}
Orbit''} 2025}) satellite systems have be come available in recent
years, however also carries the unaddressed risk of Kessler
syndrome(\protect\hyperlink{ref-KesslerSyndrome2025}{{``Kessler
Syndrome''} 2025}), where a cascading collision of satellites starts a
chain reaction leaving the entire LEO orbital space unusable for
potentially centuries.

Rural Alaskan micro-grid communities range between less than a hundred
to over 3000 people. The energy utilities in these communities are
commonly operated by a handful of individuals. Staffing rural utilities
is a challenging balance between keeping energy costs low and attracting
skilled workers.

For much of the United States, the Federal Energy Regulatory Commission
(FERC)(\protect\hyperlink{ref-HomePageFederal}{{``Home {Page} \textbar{}
{Federal Energy Regulatory Commission}''} n.d.}) is the regulator agency
that governs energy utilities in the U.S.~ ~FERC mandates that energy
utilities in the United States to follow the North American Electric
Reliability Corporation~(NERC)(\protect\hyperlink{ref-NERC}{{``{NERC}''}
n.d.}) Critical Infrastructure Protection
(CIP)(\protect\hyperlink{ref-ReliabilityStandards}{{``Reliability
{Standards}''} n.d.}) standards in regards to cybersecurity compliance
for energy utilities Operational Technology networks. However compliance
criteria are based largely on transmission capabilities. Because no
utility in Alaska is connected to the lower 48 power grid, Alaska
utilities have been effectively exempt from cybersecurity regulation.
Recently the Railbelt Reliability~Council
(RRC)(\protect\hyperlink{ref-AlaskaRailbeltReliability2025}{{``Alaska
{Railbelt Reliability Council}''} 2025}) has drafted a set of modified
CIP
standards(\protect\hyperlink{ref-CIPCriticalInfrastructure2025}{{``({CIP})
{Critical Infrastructure Protection}''} 2025}) for the State of Alaska
which are based on the NERC CIP standards but tuned to accommodate
Alaskan specific criteria. Once adopted by the Regulatory Commission of
Alaska
(RCA)(\protect\hyperlink{ref-RegulatoryCommissionAlaska}{{``Regulatory
{Commission} of {Alaska}''} n.d.}) the RRC CIP standards are expected to
become a regulator compliance requirement for those Alaskan power
producer connected to the Railbelt energy grid.

While the RRC CIP standards address the comprehensive scope of risks for
critical energy infrastructure, rural islanded Alaskan microgrids will
remain largely exempt from compliance because they do not meet the
transmission criteria. Additionally meeting cybersecurity standards
would represent a significant cost to rural communities already
struggling with the cost of energy. Not only would these communities
need to pay for expensive cybersecurity expertise, but would likely mean
expensive upgrades to existing network equipment.

\hypertarget{strategic-architecture}{%
\section{Strategic Architecture}\label{strategic-architecture}}

Clear Skies is built on a simple principle: \textbf{local-first by
design.}\\
Every system --- from the smallest sensor to the community data center
--- operates independently of the cloud services, ensuring that
essential services remain available, secure, and under local control
even when Internet connectivity is lost.

Clear Skies adopts a layered approach to build increasingly complex
modular capabilities on top of a resilient cyberinfrastructure
foundation.

\begin{figure}
\centering
\includegraphics{lib/diag/ClearSkies-Overview-notitle.excalidraw.png}
\caption{Clear Skies Reference Architecture}
\end{figure}

\hypertarget{layer-0---hardware-hw}{%
\subsection{Layer 0 - Hardware (HW)}\label{layer-0---hardware-hw}}

The hardware selection can be based on 3 tiers to accommodate different
cost, scalability, and resiliency needs.

\hypertarget{tier-1---camp-site}{%
\subsubsection{Tier 1 - Camp Site}\label{tier-1---camp-site}}

\textbf{Purpose:} Portable or training-scale deployments for small teams
and pilot projects.

\begin{itemize}
\tightlist
\item
  Commodity Grade Hardware
\item
  Low Cost of Entry and Maintenance
\item
  Portability
\item
  Limited Capacity
\item
  Basic Services
\item
  Limited Resiliency
\item
  Scales to 10's of People
\end{itemize}

\hypertarget{tier-2---village-site}{%
\subsubsection{Tier 2 - Village Site}\label{tier-2---village-site}}

\textbf{Purpose:} Fully featured, community-level cyberinfrastructure
supporting daily operations.

\begin{itemize}
\tightlist
\item
  Commodity Grade Hardware
\item
  Low Cost of Entry and Maintenance
\item
  Full Stack Service Capabilities
\item
  Full Resiliency - Zero Single Points of Failure
\item
  Scales to 100's of People
\end{itemize}

\hypertarget{tier-3---regional-site}{%
\subsubsection{Tier 3 - Regional Site}\label{tier-3---regional-site}}

\textbf{Purpose:} High-capacity, multi-community or research hub
supporting advanced services and federation.

\begin{itemize}
\tightlist
\item
  Enterprise Grade Hardware
\item
  Moderate Cost of Entry and Maintenance
\item
  Full Resiliency - Zero Single Points of Failure
\item
  Scales to 1000's of People
\end{itemize}

\hypertarget{layer-1---cyberinfrastructure-ci}{%
\subsection{Layer 1 - Cyberinfrastructure
(CI)}\label{layer-1---cyberinfrastructure-ci}}

The Cyberinfrastructure (CI) Layer forms the digital powerhouse of a
Clear Skies deployment.\\
It establishes the \textbf{core network and compute services} that allow
every community site --- from Camp Site to Regional Site --- to operate
independently of outside cloud resources.

The CI Layer is implemented as a \textbf{Software-Defined Data Center
(SDDC)}(\protect\hyperlink{ref-SoftwaredefinedDataCenter2025}{{``Software-Defined
Data Center''} 2025}): a cluster of virtualized servers that pool
compute, storage, and networking into one resilient platform.\\
This approach provides enterprise-grade reliability using open-source
tools and commodity hardware, enabling small teams to manage complex
infrastructure with minimal overhead.

\hypertarget{networking-segmentation}{%
\subsubsection{Networking \&
Segmentation}\label{networking-segmentation}}

\begin{itemize}
\tightlist
\item
  VLAN-aware switching and software-defined routing using
  \textbf{OPNsense} or similar open firewalls.
\item
  Segregated networks for Management, Operational Technology (OT), Data,
  and DMZ zones.
\item
  Local DNS, DHCP, and NTP ensuring that critical systems function
  offline.
\end{itemize}

\hypertarget{identity-trust}{%
\subsubsection{Identity \& Trust}\label{identity-trust}}

\begin{itemize}
\tightlist
\item
  \textbf{Keycloak} provides single sign-on and multi-factor
  authentication.
\item
  \textbf{Smallstep CA} or similar certificate authority issues
  short-lived internal certificates, enabling encrypted, trusted
  communication between devices and services.
\end{itemize}

\hypertarget{storage-resiliency}{%
\subsubsection{Storage \& Resiliency}\label{storage-resiliency}}

\begin{itemize}
\tightlist
\item
  \textbf{Ceph} or \textbf{ZFS-based} distributed storage replicates
  data across all nodes.
\item
  Snapshots and versioned backups protect against corruption or
  accidental deletion.\\
\item
  Air-gap or offline backup options for disaster recovery.
\end{itemize}

\hypertarget{monitoring-automation}{%
\subsubsection{Monitoring \& Automation}\label{monitoring-automation}}

\begin{itemize}
\tightlist
\item
  \textbf{Prometheus + Grafana} for metrics, alerting, and visibility.
\item
  \textbf{Ansible} or \textbf{Chef} for configuration management and
  repeatable deployments.
\item
  Logs aggregated locally via \textbf{Elastic / Wazuh / Loki} stacks.
\end{itemize}

\hypertarget{security-perimeter}{%
\subsubsection{Security \& Perimeter}\label{security-perimeter}}

\begin{itemize}
\tightlist
\item
  Dual-node firewall pairs provide high-availability failover.
\item
  Intrusion detection (Zeek/Suricata) can run as virtual appliances
  inside the same SDDC.
\item
  Role-based access control and network segmentation enforce the ``least
  privilege'' model.
\end{itemize}

\hypertarget{data-backup-synchronization}{%
\subsubsection{Data Backup \&
Synchronization}\label{data-backup-synchronization}}

\begin{itemize}
\tightlist
\item
  Automated local backups using \textbf{Restic}, \textbf{Borg}, or
  similar tools
\item
  Optional cross-site replication between Village and Regional Sites
  when connectivity permits.\\
\item
  All data remains encrypted and community-owned.
\end{itemize}

\hypertarget{layer-2---local-services-loc}{%
\subsection{Layer 2 - Local Services
(LOC)}\label{layer-2---local-services-loc}}

Layer 2 builds upon the Cyberinfrastructure (CI) foundation to deliver
the mission-specific functions that keep a community operating,
informed, and connected. These following modular service areas are
locally hosted---able to run entirely within the community network---and
can be added, removed, or upgraded without disrupting the lower layers.

Each category reflects a practical application of the local-first
philosophy: keeping critical data, control, and communication inside the
community while remaining interoperable with regional and research
partners.

\hypertarget{operational-technology-ot-scada-ics}{%
\subsubsection{Operational Technology (OT) / SCADA /
ICS}\label{operational-technology-ot-scada-ics}}

\textbf{Purpose:} maintain safe, efficient, and observable microgrid
operations under all conditions.

\begin{itemize}
\tightlist
\item
  Supervisory control and monitoring for generation, distribution, and
  storage systems.
\item
  Secure, segmented access for operators, engineers, and vendors.
\item
  Local data historians for real-time visibility even during WAN
  outages.
\item
  Integration with open-source or vendor SCADA platforms (e.g., Rapid
  SCADA, Ignition Edge, OpenPLC).
\end{itemize}

\hypertarget{industrial-internet-of-things-iiot-networks}{%
\paragraph{Industrial Internet of Things (IIoT)
Networks}\label{industrial-internet-of-things-iiot-networks}}

\textbf{Purpose:} gather and use data from across the community---power,
heat, water, environment---to inform decisions locally.

\begin{itemize}
\tightlist
\item
  LoRaWAN, Modbus TCP, and MQTT telemetry from sensors across the
  community.
\item
  Local brokers and dashboards (Node-RED, Grafana) for low-bandwidth
  visualization.
\item
  Edge analytics and rule-based automation without cloud dependence.
\end{itemize}

\hypertarget{emergency-communications}{%
\subsubsection{Emergency
Communications}\label{emergency-communications}}

\textbf{Purpose:} ensure situational awareness and coordination during
disasters or outages.

\begin{itemize}
\tightlist
\item
  Local voice, text, and alerting systems that function when commercial
  networks fail.
\item
  Interoperable with radios, satellite links, or FirstNet gateways when
  available.
\item
  Capable of community-wide paging, siren control, or automated
  messaging through existing IoT endpoints.
\end{itemize}

\hypertarget{local-community-communications}{%
\subsubsection{Local Community
Communications}\label{local-community-communications}}

\textbf{Purpose:} strengthen community cohesion and digital inclusion
through local, private communication spaces.

\begin{itemize}
\tightlist
\item
  Locally hosted chat, video, and bulletin-board tools (Matrix, Jitsi,
  etc).
\item
  Intranet portals for schools, clinics, and tribal councils.
\item
  Content caching and offline web access for education and information
  sharing.
\end{itemize}

\hypertarget{additional-service-categories-expandable}{%
\subsubsection{Additional Service Categories
(Expandable)}\label{additional-service-categories-expandable}}

\begin{itemize}
\tightlist
\item
  \textbf{Cybersecurity Operations:} IDS/IPS, log correlation,
  vulnerability scanning, and SOC visualization.
\item
  \textbf{Education \& Research Sandboxes:} student training, network
  simulation, or data-science environments.
\item
  \textbf{Local Data Services:} GIS, asset management, or archival
  storage tied to community projects.
\end{itemize}

\hypertarget{outcome}{%
\subsubsection{Outcome}\label{outcome}}

Layer 2 turns Clear Skies from infrastructure into impact --- providing
the tools that make a self-reliant community not only operationally
resilient but also informed, connected, and empowered.

\hypertarget{layer-3---community-connections-comm}{%
\subsection{Layer 3 - Community Connections
(COMM)}\label{layer-3---community-connections-comm}}

Layer 3 extends Clear Skies beyond individual communities.\\
It enables \textbf{secure collaboration, knowledge sharing, and regional
coordination} between sites --- while preserving each community's
digital sovereignty.\\
These connections are intentional, encrypted, and always under local
control. \#\#\# Secure Networking and Federation - \textbf{Tailscale /
Headscale Zero-Trust Network Access (ZTNA) Bridges:} lightweight,
encrypted overlays that connect Camp, Village, and Regional sites into a
trusted mesh without public exposure. - \textbf{Cross-Site Data
Sharing:} optional, policy-driven replication of telemetry, research,
and analytics data between communities or partner institutions. -
\textbf{Federated Identity and Trust:} local identity systems (Keycloak
/ Smallstep CA) exchange only the credentials necessary for inter-site
collaboration. - \textbf{Bandwidth-Aware Synchronization:} asynchronous,
store-and-forward file and database replication designed for limited or
intermittent connectivity.

\hypertarget{collaborative-applications}{%
\subsubsection{Collaborative
Applications}\label{collaborative-applications}}

\begin{itemize}
\tightlist
\item
  Shared monitoring dashboards and situational-awareness maps.
\item
  Federated educational resources and research datasets.
\item
  Inter-community communication tools for regional operations centers or
  cooperative utilities.
\end{itemize}

\textbf{Purpose:} build a network of sovereign digital islands --- each
self-reliant, yet capable of cooperating across Alaska's vast geography
through secure, transparent, and low-bandwidth bridges.

\hypertarget{outcome-1}{%
\subsubsection{Outcome}\label{outcome-1}}

Layer 3 transforms Clear Skies from isolated local systems into a
\textbf{distributed ecosystem of collaboration}.\\
Communities retain full control of their data and infrastructure while
participating in a resilient, Alaska-wide digital commons built on
trust, openness, and shared stewardship.

\hypertarget{technology-selection}{%
\section{Technology Selection}\label{technology-selection}}

\emph{Design and Implementation Blueprint for the Clear Skies
Architecture}

This section details the specific technologies, configurations, and
open-source components recommended for each layer and tier of the Clear
Skies architecture.\\
Selections emphasize \textbf{resilience}, \textbf{local autonomy}, and
\textbf{open interoperability} across all deployment scales.

\hypertarget{layer-0-hardware-foundations}{%
\subsection{Layer 0 --- Hardware
Foundations}\label{layer-0-hardware-foundations}}

All 3 tiers of hardware deployment can be built in a shipable rack mount
container for easy setup and portability if desired.

\begin{figure}
\centering
\includegraphics{lib/img/portable_rack_2.jpg}
\caption{Portable Rack}
\end{figure}

\hypertarget{tier-1-camp-site}{%
\subsubsection{Tier 1 --- Camp Site}\label{tier-1-camp-site}}

\emph{Portable / Training-Scale Deployment}

\begin{itemize}
\tightlist
\item
  Example hardware platforms (NUC, MiniPC, low-power servers)
\item
  Typical storage configuration (ZFS mirror, 1 GbE)
\item
  Lightweight Proxmox or single-node SDDC
\item
  Local UPS / Power considerations
\end{itemize}

\begin{figure}
\centering
\includegraphics{lib/diag/Images/VP6600_front_1600x1600.jpg}
\caption{Protectli-VP6600}
\end{figure}

\hypertarget{tier-2-village-site}{%
\subsubsection{Tier 2 --- Village Site}\label{tier-2-village-site}}

\emph{Community-Scale Deployment}

\begin{itemize}
\tightlist
\item
  Cluster of 3 × MiniPC/Protectli-class nodes
\item
  Ceph or ZFS-replicated storage
\item
  Dual OPNsense firewall HA pair
\item
  Local PoE switch with VLAN segmentation
\item
  External backup (USB or second site)
\end{itemize}

\begin{figure}
\centering
\includegraphics{lib/diag/SDDC.excalidraw.png}
\caption{Zero Single Point of Failure SDDC}
\end{figure}

\hypertarget{tier-3-regional-site}{%
\subsubsection{Tier 3 --- Regional Site}\label{tier-3-regional-site}}

\emph{Federated Multi-Community Hub}

\begin{itemize}
\tightlist
\item
  Enterprise-grade rackmount servers (ECC RAM, redundant PSU)
\item
  10 GbE backplane networking
\item
  Dedicated Ceph cluster
\item
  Multi-site replication and Tailscale/Headscale federation
\end{itemize}

\hypertarget{layer-1-cyberinfrastructure-ci}{%
\subsection{Layer 1 --- Cyberinfrastructure
(CI)}\label{layer-1-cyberinfrastructure-ci}}

\begin{itemize}
\tightlist
\item
  Virtualization Platform: \textbf{Proxmox VE / KVM}
\item
  Networking Stack: \textbf{OPNsense}, \textbf{FRR}, VLAN trunking
\item
  Storage: \textbf{Ceph}, \textbf{ZFS}, \textbf{Restic/Borg}
\item
  Identity: \textbf{Keycloak}, \textbf{Smallstep CA}
\item
  Monitoring: \textbf{Prometheus}, \textbf{Grafana}, \textbf{Loki},
  \textbf{Wazuh}
\item
  Configuration: \textbf{Ansible} or \textbf{Chef}
\end{itemize}

\hypertarget{layer-2-local-services}{%
\subsection{Layer 2 --- Local Services}\label{layer-2-local-services}}

\begin{itemize}
\tightlist
\item
  OT/SCADA: \textbf{Rapid SCADA}, \textbf{OpenPLC}, \textbf{Ignition
  Edge}
\item
  IIoT: \textbf{Mosquitto (MQTT)}, \textbf{Node-RED}, \textbf{Grafana},
  \textbf{LoRaWAN}
\item
  Comms: \textbf{Matrix (Synapse)}, \textbf{Jitsi}, \textbf{Rocket.Chat}
\item
  Cybersecurity: \textbf{Zeek}, \textbf{Suricata}, \textbf{Wazuh},
  \textbf{Elastic}
\item
  Education / Research: \textbf{JupyterHub}, \textbf{Docker / LXC
  Sandboxes}
\item
  Data: \textbf{PostgreSQL}, \textbf{GeoServer}, \textbf{Nextcloud}
\end{itemize}

\hypertarget{industrial-internet-of-thing-iiot}{%
\subsubsection{Industrial Internet of Thing
(IIoT)}\label{industrial-internet-of-thing-iiot}}

\begin{figure}
\centering
\includegraphics{lib/diag/IIoT_Architecture.excalidraw.png}
\caption{IIoT}
\end{figure}

\hypertarget{layer-3-community-connections}{%
\subsection{Layer 3 --- Community
Connections}\label{layer-3-community-connections}}

\begin{itemize}
\tightlist
\item
  Secure Networking: \textbf{Tailscale / Headscale (ZTNA Mesh)}
\item
  Federation: \textbf{Keycloak Federation}, \textbf{Smallstep
  cross-trust}
\item
  Data Sync: \textbf{Syncthing}, \textbf{rsync}, \textbf{MinIO Gateway}
\item
  Shared Visualization: \textbf{Grafana Federation}, \textbf{Kibana
  Dashboards}
\item
  Optional: Integration with \textbf{FirstNet}, \textbf{Starlink}, or
  terrestrial backhaul for redundancy
\end{itemize}

\hypertarget{hardware-specifications}{%
\section{Hardware Specifications}\label{hardware-specifications}}

\begin{dangerbox}[Vendor Agnostic]
As it is a stated goal of the Clear Skies architecture to remain vendor
agnostic the following vendor product highlights are for comparison
purposes only and not recommendations or promotions an any specific
vendor or products.
\end{dangerbox}

\hypertarget{sddc-servers}{%
\subsection{SDDC Servers}\label{sddc-servers}}

The following data compared serveral technical and capacity aspects of
potential hardware solutions for a Software Defined Data Center (SDDC)
ProxMox server / PVE node.

\hypertarget{proxmox-ve-server-hardware-requirements}{%
\subsubsection{Proxmox VE Server Hardware
Requirements}\label{proxmox-ve-server-hardware-requirements}}

\scriptsize

\begin{longtable}[]{@{}
  >{\raggedright\arraybackslash}p{(\columnwidth - 6\tabcolsep) * \real{0.1299}}
  >{\raggedright\arraybackslash}p{(\columnwidth - 6\tabcolsep) * \real{0.2429}}
  >{\raggedright\arraybackslash}p{(\columnwidth - 6\tabcolsep) * \real{0.2599}}
  >{\raggedright\arraybackslash}p{(\columnwidth - 6\tabcolsep) * \real{0.3672}}@{}}
\toprule
\begin{minipage}[b]{\linewidth}\raggedright
\textbf{Category}
\end{minipage} & \begin{minipage}[b]{\linewidth}\raggedright
\textbf{Bare Minimum} (Lab/Test)
\end{minipage} & \begin{minipage}[b]{\linewidth}\raggedright
\textbf{Standalone} (Edge)
\end{minipage} & \begin{minipage}[b]{\linewidth}\raggedright
\textbf{Hyperconverged Node} (Cluster)
\end{minipage} \\
\midrule
\endhead
\textbf{CPU} & 1× Dual-Core (Intel/AMD, VT-x/AMD-V) & 1× Quad-Core
(i5/i7, Xeon-E, Ryzen 5/7) & 1× 6--12 Core (Xeon-D, Xeon-Silver, Ryzen
9, EPYC) \\
\textbf{Architecture} & x86-64 & x86-64 & x86-64 (SR-IOV \& AES-NI
support recommended) \\
\textbf{RAM} & 8 GB minimum (test only) & 32--64 GB & 64--256 GB (ECC
prefered) \\
\textbf{Boot / OS Disk} & 64 GB SATA SSD & 128 GB SATA/NVMe SSD & 256 GB
NVMe SSD (mirrored or ZFS mirror) \\
\textbf{VM/CT Storage} & \textgreater\textasciitilde{} 250 GB SSD/HDD &
\textgreater\textasciitilde{} 1 TB Single SSD/NVMe & 2+
\textgreater\textasciitilde2 TB NVMe/SSD \\
\textbf{Network Interfaces} & 2× 1 GbE & 3× 1/2.5 GbE (LAN, WAN, Mgmt) &
4--6× 2.5/10 GbE (Mgmt, Ceph, VM LAN, Public, Storage) \\
\textbf{Out-of-Band Mgmt} & Optional & Optional & Recommended (IPMI,
iDRAC, Etc) \\
\textbf{Power Supply} & Single PSU & Single PSU & Recommended Dual
hot-swappable PSUs \\
\textbf{TPM / Secure Boot} & Optional & Recommended & Required for
Microsoft compliance (TPM 2.0) \\
\textbf{BIOS / Firmware} & Legacy or UEFI & UEFI (coreboot OK) & UEFI \\
\textbf{Cluster / Ceph Role} & N/A & Optional (single node) & Full
cluster member (Ceph OSD + Monitor) \\
\textbf{Performance Target} & Small lab / field site & Small-scale
production workloads & Continuous 24×7 ops with fault tolerance \\
\textbf{Approx Power Draw} & 25--40 W & 50--90 W & 80--200 W (depending
on drives/NICs) \\
\textbf{Example Platform} & Intel NUC, Protectli VP6630 & Minisforum
MS-01, Protectli VP6650 & Supermicro E300, Xeon-D, or 3× Proxmox
mini-cluster \\
\textbf{Notes} & Not for production & Great for edge compute or small
SDDC & Use 3 nodes + Ceph + replication; no single failure halts
cluster \\
\bottomrule
\end{longtable}

\hypertarget{server-comparison}{%
\subsection{Server Comparison}\label{server-comparison}}

\scriptsize

\begin{longtable}[]{@{}
  >{\raggedright\arraybackslash}p{(\columnwidth - 18\tabcolsep) * \real{0.0833}}
  >{\raggedright\arraybackslash}p{(\columnwidth - 18\tabcolsep) * \real{0.0833}}
  >{\centering\arraybackslash}p{(\columnwidth - 18\tabcolsep) * \real{0.1111}}
  >{\raggedright\arraybackslash}p{(\columnwidth - 18\tabcolsep) * \real{0.0833}}
  >{\raggedright\arraybackslash}p{(\columnwidth - 18\tabcolsep) * \real{0.0833}}
  >{\centering\arraybackslash}p{(\columnwidth - 18\tabcolsep) * \real{0.1111}}
  >{\centering\arraybackslash}p{(\columnwidth - 18\tabcolsep) * \real{0.1111}}
  >{\centering\arraybackslash}p{(\columnwidth - 18\tabcolsep) * \real{0.1111}}
  >{\centering\arraybackslash}p{(\columnwidth - 18\tabcolsep) * \real{0.1111}}
  >{\centering\arraybackslash}p{(\columnwidth - 18\tabcolsep) * \real{0.1111}}@{}}
\toprule
\begin{minipage}[b]{\linewidth}\raggedright
Product
\end{minipage} & \begin{minipage}[b]{\linewidth}\raggedright
CPU (Make + Cores)
\end{minipage} & \begin{minipage}[b]{\linewidth}\centering
RAM (GB)
\end{minipage} & \begin{minipage}[b]{\linewidth}\raggedright
OS Disk
\end{minipage} & \begin{minipage}[b]{\linewidth}\raggedright
VM Disk(s)
\end{minipage} & \begin{minipage}[b]{\linewidth}\centering
1--2 Gb NICs
\end{minipage} & \begin{minipage}[b]{\linewidth}\centering
10 Gb NICs
\end{minipage} & \begin{minipage}[b]{\linewidth}\centering
Rack (U)
\end{minipage} & \begin{minipage}[b]{\linewidth}\centering
Power (W max)
\end{minipage} & \begin{minipage}[b]{\linewidth}\centering
Price (USD)
\end{minipage} \\
\midrule
\endhead
Qotom Q30900GE S13 Series & Intel 8th/10th Gen (2C) & 32 & 2.5-inch SATA
SSD/HDD 0 TB & Mini PCIe mSATA SSD x1 0 TB; M.2 Wi-Fi E-Key (2230) x1 0
TB & 8 & 0 & 1U & 30 & \$489 \\
MINIS FORUM MS-A2 & AMD Ryzen 9 9955HX (16C) & 96 & M.2 2280/U.2 NVMe
SSD 2 TB & M.2 2280/22110 NVMe SSD x1 0 TB & 2 & 2 & 2U (approx.) & 130
& \$1495.9 \\
Protectli VP6630 & Intel Core i3 (4C) & 96 & NVMe SSD 4 TB & SATA SSD x1
1 TB & 6 & 2 & 1U & 40 & \$1651 \\
Protectli VP6650 & Intel Core i5 (4C) & 96 & NVMe SSD 4 TB & SATA SSD x1
1 TB & 6 & 2 & 1U & 45 & \$1811 \\
MINIS FORUM MS-S1 Max & AMD Ryzen (16C) & 128 & NVMe SSD 2 TB & -- & 0 &
2 & 2U & 130 & \$2503.9 \\
Lancelot 1199-SR & Intel Xeon (8C) & 128 & NVMe SSD 1 TB & SAS HDD x4 16
TB & 2 & 4 & 1U & 250 & \$5199 \\
ProLiant DL145 Gen11 & AMD EPYC 8124P (16C) & 128 & SATA SSD 0.96 TB &
SATA SSD x1 3.84 TB & 2 & 0 & 1U & 350 & \$11250.0 \\
Lancelot 1898-N12 & Intel Xeon Silver 4514Y (32C) & 256 & NVMe SSD 1.0
TB & NVMe SSD x2 15.36 TB & 0 & 6 & 1U & 600 & \$11727.0 \\
ProLiant DL325 Gen11 & AMD EPYC 9124 (16C) & 128 & SATA SSD 3.84 TB &
SATA SSD x2 -- TB & 2 & 0 & 1U & 400 & \$16231.82 \\
PowerEdge R6615 & AMD EPYC 9224 (24C) & 96 & SATA SSD 0.96 TB & SATA SSD
x4 3.84 TB & 2 & 2 & 1U & 450 & \$19401.16 \\
\bottomrule
\end{longtable}

\small

\hypertarget{server-specifications}{%
\subsection{Server Specifications}\label{server-specifications}}

\hypertarget{qotom-q30900ge-s13-series}{%
\subsubsection{Qotom Q30900GE S13
Series}\label{qotom-q30900ge-s13-series}}

\begin{figure}
\centering
\includegraphics{lib/img/Qotom_Q30900se-s13.jpg}
\caption{Qotom Q30900GE S13 Series}
\end{figure}

\textbf{Specifications}

\begin{longtable}[]{@{}
  >{\raggedright\arraybackslash}p{(\columnwidth - 2\tabcolsep) * \real{0.5000}}
  >{\raggedright\arraybackslash}p{(\columnwidth - 2\tabcolsep) * \real{0.5000}}@{}}
\toprule
\begin{minipage}[b]{\linewidth}\raggedright
Spec
\end{minipage} & \begin{minipage}[b]{\linewidth}\raggedright
Value
\end{minipage} \\
\midrule
\endhead
CPU & Intel 8th/10th Gen (2 cores, 4 threads) \\
Memory & 32 GB DDR4 SO-DIMM 2133/2400 MHz \\
OS Disk & 2.5-inch SATA SSD/HDD 0 TB \\
VM Disk(s) & Mini PCIe mSATA SSD x1 0 TB; M.2 Wi-Fi E-Key (2230) x1 0
TB \\
1--2 Gb NICs & 8 \\
10 Gb NICs & 0 \\
Rack Units & 1U \\
Dimensions (in) & \{`l': 7.7, `w': 4.8, `h': 1.9\} \\
Power Draw (W) & Idle 10 / Max 30 \\
Power Input & DC 12 V Jack (5.5 mm × 2.5 mm) \\
Management & BMC: False, BIOS: UEFI \\
Supported OS & Windows 10, Linux (Ubuntu 24.04 LTS, Proxmox VE,
OPNsense) \\
Price (USD) & \$489 \\
Product Page &
\href{https://www.qotom.net/product/MiniPC_Q30900SE_S13_Series.html}{Link} \\
\bottomrule
\end{longtable}

\hypertarget{minis-forum-ms-a2}{%
\subsubsection{MINIS FORUM MS-A2}\label{minis-forum-ms-a2}}

\begin{figure}
\centering
\includegraphics{lib/img/Minisforum_MS-A2.jpg}
\caption{MINIS FORUM MS-A2}
\end{figure}

\textbf{Specifications}

\begin{longtable}[]{@{}
  >{\raggedright\arraybackslash}p{(\columnwidth - 2\tabcolsep) * \real{0.5000}}
  >{\raggedright\arraybackslash}p{(\columnwidth - 2\tabcolsep) * \real{0.5000}}@{}}
\toprule
\begin{minipage}[b]{\linewidth}\raggedright
Spec
\end{minipage} & \begin{minipage}[b]{\linewidth}\raggedright
Value
\end{minipage} \\
\midrule
\endhead
CPU & AMD Ryzen 9 9955HX (16 cores, 32 threads) \\
Memory & 96 GB DDR5 SO-DIMM 5600 MHz \\
OS Disk & M.2 2280/U.2 NVMe SSD 2 TB \\
VM Disk(s) & M.2 2280/22110 NVMe SSD x1 0 TB \\
1--2 Gb NICs & 2 \\
10 Gb NICs & 2 \\
Rack Units & 2U (approx.) \\
Dimensions (in) & \{`l': 7.7, `w': 7.4, `h': 1.9\} \\
Power Draw (W) & Idle 35 / Max 130 \\
Power Input & DC 19V/12.63A Adapter \\
Management & BMC: False, BIOS: UEFI Secure Boot \\
Supported OS & Windows 11, Linux (Ubuntu 24.04 LTS, Proxmox VE) \\
Price (USD) & \$1495.9 \\
Product Page &
\href{https://store.minisforum.com/products/minisforum-ms-a2?variant=46843404943605}{Link} \\
\bottomrule
\end{longtable}

\hypertarget{protectli-vp6630}{%
\subsubsection{Protectli VP6630}\label{protectli-vp6630}}

\begin{figure}
\centering
\includegraphics{lib/img/Protectli_VP6630.jpg}
\caption{Protectli VP6630}
\end{figure}

\textbf{Specifications}

\begin{longtable}[]{@{}ll@{}}
\toprule
Spec & Value \\
\midrule
\endhead
CPU & Intel Core i3 (4 cores, 8 threads) \\
Memory & 96 GB 2x48GB DDR5-5600 SO-DIMM \\
OS Disk & NVMe SSD 4 TB \\
VM Disk(s) & SATA SSD x1 1 TB \\
1--2 Gb NICs & 6 \\
10 Gb NICs & 2 \\
Rack Units & 1U \\
Dimensions (in) & \{`l': 7.5, `w': 7.0, `h': 3.0\} \\
Power Draw (W) & Idle 15 / Max 40 \\
Power Input & DC 19.5V (IEC Type B) \\
Management & BMC: False, BIOS: coreboot (Open Source) \\
Supported OS & Proxmox VE, OPNsense, Ubuntu 24.04 LTS \\
Price (USD) & \$1651 \\
Product Page & \href{https://protectli.com/product/vp6630/}{Link} \\
\bottomrule
\end{longtable}

\hypertarget{protectli-vp6650}{%
\subsubsection{Protectli VP6650}\label{protectli-vp6650}}

\begin{figure}
\centering
\includegraphics{lib/img/Protectli_VP6650.jpg}
\caption{Protectli VP6650}
\end{figure}

\textbf{Specifications}

\begin{longtable}[]{@{}ll@{}}
\toprule
Spec & Value \\
\midrule
\endhead
CPU & Intel Core i5 (4 cores, 8 threads) \\
Memory & 96 GB 2x48GB DDR5-5600 SO-DIMM \\
OS Disk & NVMe SSD 4 TB \\
VM Disk(s) & SATA SSD x1 1 TB \\
1--2 Gb NICs & 6 \\
10 Gb NICs & 2 \\
Rack Units & 1U \\
Dimensions (in) & \{`l': 7.5, `w': 7.0, `h': 3.0\} \\
Power Draw (W) & Idle 15 / Max 45 \\
Power Input & DC 19.5V (IEC Type B) \\
Management & BMC: False, BIOS: coreboot \\
Supported OS & Proxmox VE, OPNsense, Ubuntu 24.04 LTS \\
Price (USD) & \$1811 \\
Product Page & \href{https://protectli.com/product/vp6650/}{Link} \\
\bottomrule
\end{longtable}

\hypertarget{minis-forum-ms-s1-max}{%
\subsubsection{MINIS FORUM MS-S1 Max}\label{minis-forum-ms-s1-max}}

\begin{figure}
\centering
\includegraphics{lib/img/Minisforum_MS-S1-Max.jpg}
\caption{MINIS FORUM MS-S1 Max}
\end{figure}

\textbf{Specifications}

\begin{longtable}[]{@{}
  >{\raggedright\arraybackslash}p{(\columnwidth - 2\tabcolsep) * \real{0.5000}}
  >{\raggedright\arraybackslash}p{(\columnwidth - 2\tabcolsep) * \real{0.5000}}@{}}
\toprule
\begin{minipage}[b]{\linewidth}\raggedright
Spec
\end{minipage} & \begin{minipage}[b]{\linewidth}\raggedright
Value
\end{minipage} \\
\midrule
\endhead
CPU & AMD Ryzen (16 cores, 32 threads) \\
Memory & 128 GB LPDDR5x-8000 \\
OS Disk & NVMe SSD 2 TB \\
VM Disk(s) & -- \\
1--2 Gb NICs & 0 \\
10 Gb NICs & 2 \\
Rack Units & 2U \\
Dimensions (in) & \{`l': 8.7, `w': 8.1, `h': 3.0\} \\
Power Draw (W) & Idle 35 / Max 130 \\
Power Input & DC 12V/26.6A (320W adapter) \\
Management & BMC: False, BIOS: UEFI Secure Boot \\
Supported OS & Windows 11 Pro, Ubuntu 24.04 LTS, Proxmox VE \\
Price (USD) & \$2503.9 \\
Product Page &
\href{https://store.minisforum.com/products/minisforum-ms-s1-max-mini-pc}{Link} \\
\bottomrule
\end{longtable}

\hypertarget{lancelot-1199-sr}{%
\subsubsection{Lancelot 1199-SR}\label{lancelot-1199-sr}}

\begin{figure}
\centering
\includegraphics{lib/img/ASL_Lancelot_1199SR.jpg}
\caption{Lancelot 1199-SR}
\end{figure}

\textbf{Specifications}

\begin{longtable}[]{@{}
  >{\raggedright\arraybackslash}p{(\columnwidth - 2\tabcolsep) * \real{0.5000}}
  >{\raggedright\arraybackslash}p{(\columnwidth - 2\tabcolsep) * \real{0.5000}}@{}}
\toprule
\begin{minipage}[b]{\linewidth}\raggedright
Spec
\end{minipage} & \begin{minipage}[b]{\linewidth}\raggedright
Value
\end{minipage} \\
\midrule
\endhead
CPU & Intel Xeon (8 cores, 16 threads) \\
Memory & 128 GB DDR5-4800 ECC RDIMM \\
OS Disk & NVMe SSD 1 TB \\
VM Disk(s) & SAS HDD x4 16 TB \\
1--2 Gb NICs & 2 \\
10 Gb NICs & 4 \\
Rack Units & 1U \\
Dimensions (in) & \{`l': 25.6, `w': 17.2, `h': 1.7\} \\
Power Draw (W) & Idle 95 / Max 250 \\
Power Input & AC 100--240V \\
Management & BMC: True, BIOS: UEFI with BMC (AST2600) \\
Supported OS & Proxmox VE, Ubuntu 24.04 LTS, Ceph \\
Price (USD) & \$5199 \\
Product Page &
\href{https://www.aslab.com/products/rackmount/customize/lancelot1199sr.cgi}{Link} \\
\bottomrule
\end{longtable}

\hypertarget{proliant-dl145-gen11}{%
\subsubsection{ProLiant DL145 Gen11}\label{proliant-dl145-gen11}}

\begin{figure}
\centering
\includegraphics{lib/img/HP_ProLiant_DL145_Gen11.jpg}
\caption{ProLiant DL145 Gen11}
\end{figure}

\textbf{Specifications}

\begin{longtable}[]{@{}
  >{\raggedright\arraybackslash}p{(\columnwidth - 2\tabcolsep) * \real{0.5000}}
  >{\raggedright\arraybackslash}p{(\columnwidth - 2\tabcolsep) * \real{0.5000}}@{}}
\toprule
\begin{minipage}[b]{\linewidth}\raggedright
Spec
\end{minipage} & \begin{minipage}[b]{\linewidth}\raggedright
Value
\end{minipage} \\
\midrule
\endhead
CPU & AMD EPYC 8124P (16 cores, 32 threads) \\
Memory & 128 GB DDR5-4800 ECC Registered (HPE SmartMemory) \\
OS Disk & SATA SSD 0.96 TB \\
VM Disk(s) & SATA SSD x1 3.84 TB \\
1--2 Gb NICs & 2 \\
10 Gb NICs & 0 \\
Rack Units & 1U \\
Dimensions (in) & \{`l': 27.5, `w': 17.5, `h': 1.7\} \\
Power Draw (W) & Idle 95 / Max 350 \\
Power Input & AC 100--240V \\
Management & BMC: True, BIOS: UEFI / iLO 6 \\
Supported OS & Proxmox VE, Ubuntu 24.04 LTS, Red Hat Enterprise Linux 9,
Windows Server 2025 \\
Price (USD) & \$11250.0 \\
Product Page &
\href{https://buy.hpe.com/us/en/compute/rack-servers/proliant-dl100-servers/proliant-dl145-server/hpe-proliant-dl145-gen11/p/1014845266}{Link} \\
\bottomrule
\end{longtable}

\hypertarget{lancelot-1898-n12}{%
\subsubsection{Lancelot 1898-N12}\label{lancelot-1898-n12}}

\begin{figure}
\centering
\includegraphics{lib/img/ASL_Lancelot_1898-N12.jpg}
\caption{Lancelot 1898-N12}
\end{figure}

\textbf{Specifications}

\begin{longtable}[]{@{}
  >{\raggedright\arraybackslash}p{(\columnwidth - 2\tabcolsep) * \real{0.5000}}
  >{\raggedright\arraybackslash}p{(\columnwidth - 2\tabcolsep) * \real{0.5000}}@{}}
\toprule
\begin{minipage}[b]{\linewidth}\raggedright
Spec
\end{minipage} & \begin{minipage}[b]{\linewidth}\raggedright
Value
\end{minipage} \\
\midrule
\endhead
CPU & Intel Xeon Silver 4514Y (32 cores, 64 threads) \\
Memory & 256 GB DDR5-4800 ECC Registered \\
OS Disk & NVMe SSD 1.0 TB \\
VM Disk(s) & NVMe SSD x2 15.36 TB \\
1--2 Gb NICs & 0 \\
10 Gb NICs & 6 \\
Rack Units & 1U \\
Dimensions (in) & \{`l': 28.0, `w': 17.6, `h': 3.4\} \\
Power Draw (W) & Idle 180 / Max 600 \\
Power Input & AC 100--240V \\
Management & BMC: True, BIOS: UEFI / ASPEED AST2600 \\
Supported OS & Rocky Linux 8.10, Red Hat Enterprise Linux 8.10, Ubuntu
24.04 LTS, Proxmox VE 8 \\
Price (USD) & \$11727.0 \\
Product Page &
\href{https://www.asus.com/Server-Workstation/Servers/1U-2U-Rack-Servers/RS700-E11-RS12U/}{Link} \\
\bottomrule
\end{longtable}

\hypertarget{proliant-dl325-gen11}{%
\subsubsection{ProLiant DL325 Gen11}\label{proliant-dl325-gen11}}

\begin{figure}
\centering
\includegraphics{lib/img/HP_ProLiant_DL235_Gen11.jpg}
\caption{ProLiant DL325 Gen11}
\end{figure}

\textbf{Specifications}

\begin{longtable}[]{@{}
  >{\raggedright\arraybackslash}p{(\columnwidth - 2\tabcolsep) * \real{0.5000}}
  >{\raggedright\arraybackslash}p{(\columnwidth - 2\tabcolsep) * \real{0.5000}}@{}}
\toprule
\begin{minipage}[b]{\linewidth}\raggedright
Spec
\end{minipage} & \begin{minipage}[b]{\linewidth}\raggedright
Value
\end{minipage} \\
\midrule
\endhead
CPU & AMD EPYC 9124 (16 cores, 32 threads) \\
Memory & 128 GB DDR5-4800 ECC Registered (HPE SmartMemory) \\
OS Disk & SATA SSD 3.84 TB \\
VM Disk(s) & SATA SSD x2 -- TB \\
1--2 Gb NICs & 2 \\
10 Gb NICs & 0 \\
Rack Units & 1U \\
Dimensions (in) & \{`l': 28.0, `w': 17.5, `h': 1.7\} \\
Power Draw (W) & Idle 110 / Max 400 \\
Power Input & AC 100--240V \\
Management & BMC: True, BIOS: UEFI / iLO 6 \\
Supported OS & Proxmox VE, Ubuntu 24.04 LTS, Red Hat Enterprise Linux 9,
Windows Server 2025 \\
Price (USD) & \$16231.82 \\
Product Page &
\href{https://buy.hpe.com/us/en/compute/rack-servers/proliant-dl300-servers/proliant-dl325-server/hpe-proliant-dl325-gen11/p/1014689141}{Link} \\
\bottomrule
\end{longtable}

\hypertarget{poweredge-r6615}{%
\subsubsection{PowerEdge R6615}\label{poweredge-r6615}}

\begin{figure}
\centering
\includegraphics{lib/img/Dell_PowerEdge_R6615.jpg}
\caption{PowerEdge R6615}
\end{figure}

\textbf{Specifications}

\begin{longtable}[]{@{}
  >{\raggedright\arraybackslash}p{(\columnwidth - 2\tabcolsep) * \real{0.5000}}
  >{\raggedright\arraybackslash}p{(\columnwidth - 2\tabcolsep) * \real{0.5000}}@{}}
\toprule
\begin{minipage}[b]{\linewidth}\raggedright
Spec
\end{minipage} & \begin{minipage}[b]{\linewidth}\raggedright
Value
\end{minipage} \\
\midrule
\endhead
CPU & AMD EPYC 9224 (24 cores, 48 threads) \\
Memory & 96 GB DDR5-5600 ECC RDIMM \\
OS Disk & SATA SSD 0.96 TB \\
VM Disk(s) & SATA SSD x4 3.84 TB \\
1--2 Gb NICs & 2 \\
10 Gb NICs & 2 \\
Rack Units & 1U \\
Dimensions (in) & \{`l': 28.0, `w': 17.1, `h': 1.7\} \\
Power Draw (W) & Idle 120 / Max 450 \\
Power Input & AC 100--240V \\
Management & BMC: True, BIOS: UEFI / iDRAC9 Express 16G \\
Supported OS & Proxmox VE, Ubuntu Server 24.04 LTS, Red Hat Enterprise
Linux 9, Windows Server 2025 \\
Price (USD) & \$19401.16 \\
Product Page &
\href{https://www.dell.com/en-us/shop/cty/pdp/spd/poweredge-r6615/pe_r6615_tm_vi_vp_sb?configurationid=1759700b-2877-411f-bf22-461cea367d8e}{Link} \\
\bottomrule
\end{longtable}

\hypertarget{terminology}{%
\section{Terminology}\label{terminology}}

\begin{longtable}[]{@{}
  >{\raggedright\arraybackslash}p{(\columnwidth - 4\tabcolsep) * \real{0.1179}}
  >{\raggedright\arraybackslash}p{(\columnwidth - 4\tabcolsep) * \real{0.2051}}
  >{\raggedright\arraybackslash}p{(\columnwidth - 4\tabcolsep) * \real{0.6769}}@{}}
\toprule
\begin{minipage}[b]{\linewidth}\raggedright
Acronym
\end{minipage} & \begin{minipage}[b]{\linewidth}\raggedright
Term
\end{minipage} & \begin{minipage}[b]{\linewidth}\raggedright
Description
\end{minipage} \\
\midrule
\endhead
\textbf{AC} & Alternating Current & \textasciitilde60 Hz 120 Volt power
with an oscillating voltage. \\
\textbf{ACEP} & Alaska Center for Energy and Power & University of
Alaska Fairbanks research center focused on applied energy systems and
innovation in rural and microgrid environments. \\
\textbf{CA} & Certificate Authority & Service that issues and manages
digital certificates used to authenticate and encrypt communications. \\
\textbf{Ceph} & --- & Open-source distributed storage system providing
block, object, and file storage across clustered nodes. \\
\textbf{CI} & Cyberinfrastructure & The foundational compute, storage,
and network systems enabling digital services to operate locally and
independently. \\
\textbf{DC} & Direct Current & Contant Voltage Power Systems such as
provided by batteries. \\
\textbf{DMZ} & Demilitarized Zone & Network segment that isolates
external-facing systems from internal critical infrastructure. \\
\textbf{DNS} & Domain Name System & Converts human-readable hostnames
into IP addresses. \\
\textbf{DHCP} & Dynamic Host Configuration Protocol & Automatically
assigns IP addresses to devices on a network. \\
\textbf{HW} & Hardware & Physical computing, storage, and network
devices forming the foundation of the infrastructure. \\
\textbf{ICS} & Industrial Control System & Hardware and software used to
monitor and control industrial processes such as generation and
distribution. \\
\textbf{IIoT} & Industrial Internet of Things & Networked sensors and
devices that collect and exchange data for monitoring and automation in
industrial settings. \\
\textbf{LAN} & Local Area Network & Internal network connecting devices
within a limited geographic area such as a facility or village. \\
\textbf{LLM} & Large Language Model & AI model trained on vast text
corpora to generate and analyze natural language. Used locally for
automation and data analysis. \\
\textbf{LOC} & Local Services Layer & Layer 2 in the Clear Skies
architecture providing operational, communication, and data services
within the community. \\
\textbf{MQTT} & Message Queuing Telemetry Transport & Lightweight
publish/subscribe messaging protocol optimized for low-bandwidth IIoT
networks. \\
\textbf{NTP} & Network Time Protocol & Synchronizes system clocks across
devices on a network. \\
\textbf{OPNsense} & --- & Open-source firewall and routing platform
providing VLAN segmentation, VPNs, and intrusion detection. \\
\textbf{OT} & Operational Technology & Systems that monitor and control
physical devices, processes, and infrastructure. \\
\textbf{PLC} & Programmable Logic Controller & Industrial computer used
to automate electromechanical processes. \\
\textbf{PVE} & ProxMox Virtual Environment & Open-source virtualization
environment used to create Software-Defined Data Centers (SDDC). \\
\textbf{PSU} & Power Supply Unit & A hot swapable power supply in a rack
mount server or other equipment. \\
\textbf{SCADA} & Supervisory Control and Data Acquisition & System for
remote monitoring and control of industrial and utility operations. \\
\textbf{SDDC} & Software-Defined Data Center & Virtualized data center
architecture where compute, storage, and networking are abstracted from
hardware. \\
\textbf{SDN} & Software-Defined Networking & Network architecture
enabling centralized, programmable control of traffic and
segmentation. \\
\textbf{SOC} & Security Operations Center & Centralized facility or
function for monitoring, detecting, and responding to cybersecurity
threats. \\
\textbf{Tailscale Headscale} & --- & Zero-trust networking tools that
establish secure, peer-to-peer mesh connectivity across sites. \\
\textbf{UPS} & Uninterruptable Power Supply & A batter backup DC to AC
inverter system to provide AC power during intermittent short duration
power outages. \\
\textbf{ZTNA} & Zero Trust Network Access & Security framework that
assumes no implicit trust and enforces strict identity-based access
controls for every connection. \\
\bottomrule
\end{longtable}

\hypertarget{citations}{%
\section*{Citations}\label{citations}}
\addcontentsline{toc}{section}{Citations}

\hypertarget{refs}{}
\begin{CSLReferences}{1}{0}
\leavevmode\vadjust pre{\hypertarget{ref-AlaskaRailbeltReliability2025}{}}%
{``Alaska {Railbelt Reliability Council}.''} 2025. \emph{RRC Local}.
https://www.akrrc.org/.

\leavevmode\vadjust pre{\hypertarget{ref-CIPCriticalInfrastructure2025}{}}%
{``({CIP}) {Critical Infrastructure Protection}.''} 2025. \emph{RRC
Local}.
https://www.akrrc.org/matters/category/cip-critical-infrastructure-protection.

\leavevmode\vadjust pre{\hypertarget{ref-HomePageFederal}{}}%
{``Home {Page} \textbar{} {Federal Energy Regulatory Commission}.''}
n.d. https://www.ferc.gov/. Accessed November 7, 2025.

\leavevmode\vadjust pre{\hypertarget{ref-KesslerSyndrome2025}{}}%
{``Kessler Syndrome.''} 2025. \emph{Wikipedia}, October.

\leavevmode\vadjust pre{\hypertarget{ref-LowEarthOrbit2025}{}}%
{``Low {Earth} Orbit.''} 2025. \emph{Wikipedia}, October.

\leavevmode\vadjust pre{\hypertarget{ref-NERC}{}}%
{``{NERC}.''} n.d. https://www.nerc.com/Pages/default.aspx. Accessed
November 7, 2025.

\leavevmode\vadjust pre{\hypertarget{ref-RegulatoryCommissionAlaska}{}}%
{``Regulatory {Commission} of {Alaska}.''} n.d.
https://rca.alaska.gov/RCAWeb/home.aspx. Accessed November 7, 2025.

\leavevmode\vadjust pre{\hypertarget{ref-ReliabilityStandards}{}}%
{``Reliability {Standards}.''} n.d.
https://www.nerc.com/pa/Stand/Pages/ReliabilityStandards.aspx. Accessed
November 7, 2025.

\leavevmode\vadjust pre{\hypertarget{ref-SoftwaredefinedDataCenter2025}{}}%
{``Software-Defined Data Center.''} 2025. \emph{Wikipedia}, September.

\end{CSLReferences}

\end{document}
